%%%%%%%%%%%%%%%%%%%%%%%%%%%%%%%%%%%%%%%%%%%%%%%%%%%%%%%%%%%
% --------------------------------------------------------
% Tau
% LaTeX Template
% Version 2.4.4 (28/02/2025)
%
% Author: 
% Guillermo Jimenez (memo.notess1@gmail.com)
% 
% License:
% Creative Commons CC BY 4.0
% --------------------------------------------------------
%%%%%%%%%%%%%%%%%%%%%%%%%%%%%%%%%%%%%%%%%%%%%%%%%%%%%%%%%%%

\documentclass[9pt,a4paper,twocolumn,twoside]{tau-class/tau}
\usepackage[english]{babel}

%% Spanish babel recomendation
% \usepackage[spanish,es-nodecimaldot,es-noindentfirst]{babel} 

%% Draft watermark
% \usepackage{draftwatermark}

%----------------------------------------------------------
% TITLE
%----------------------------------------------------------

\journalname{Example Template}
\title{Writing a lab report or academic article with tau \LaTeX\ class}

%----------------------------------------------------------
% AUTHORS, AFFILIATIONS AND PROFESSOR
%----------------------------------------------------------

\author[a,1]{Author One}
\author[b,2]{Author Two}
\author[b,c,3]{Author Three}

%----------------------------------------------------------

\affil[a]{Affiliation of author one}
\affil[b]{Affiliation of author two}
\affil[c]{Affiliation of author three}

\professor{Professor/Authority or other information}

%----------------------------------------------------------
% FOOTER INFORMATION
%----------------------------------------------------------

\institution{College name}
\footinfo{\LaTeX\ Template}
\theday{July 26, 2024}
\leadauthor{Author last name et al.}
\course{Creative Commons CC BY 4.0}

%----------------------------------------------------------
% ABSTRACT AND KEYWORDS
%----------------------------------------------------------

\begin{abstract}    
    Welcome to tau ($\tau$) \LaTeX\ class designed especially for your lab reports or academic articles. In this example template, we will guide you through the process of using and customizing this document to your needs. For more information of this class check out the appendix section. There, you will find codes that define key aspects of the template, allowing you to explore and modify them.
\end{abstract}

%----------------------------------------------------------

\keywords{\LaTeX\ class, lab report, academic article, tau class}

%----------------------------------------------------------

\begin{document}
		
    \maketitle 
    \thispagestyle{firststyle} 
    \tauabstract 
    % \tableofcontents
    % \linenumbers 
    
%----------------------------------------------------------

\section{Introduction}

    \taustart{W}elcome to \textit{tau class} for preparing your lab reports or academic articles. Throughout this guide, we will show you how to use this template and how to make modifications to this class. 
	
    This class includes the following files placed in the ‘tau-class’ folder: tau.cls, tauenvs.sty, taubabel.sty and README.md. Also, a main.tex, tau.bib and some examples. 

\section{Title}

    The \verb*|\maketitle| command generates the title and author information section, including the professor name and affiliations. The title can be modified in tau-class/tau.cls/title style section. 
	
    By default, \textit{tau class} shows the title on the left. However, you can change \verb*|\raggedright| to \verb*|\centering| in \verb*|\titlepos| to move the title to the center or, modify it to your own preferences.
	
    In addition to the \verb|\title| command, a custom command named \verb|\journalname| has been added to include more information. 
	
    If you do not need this command, you can undefined it and the content will be adjusted automatically.
	
\section{Abstract}

    The abstract and keywords are defined using the \verb*|\keywords| and \verb*|\begin{abstract} \end{abstract}| commands respectively. For the abstract to appear, make sure the \verb|\tauabstract| command is always included after the beginning of the document.
    
    If the keywords are not declared in the preamble, the content will be adjusted automatically.
    
\section{Document style options}

    \subsection{Tau start}
	
        We included the \verb|\taustart{}| command, which provides a personalized lettrine for the beginning of a paragraph.

    \subsection{Line numbering}
	
        By implementing the \textit{lineno} package, the line numbering of the document can be placed with the command \verb|\linenumbers|. 
		
        I recommend placing the command after the abstract and table of contents for a better appearance.
		
    \subsection{Table of contents}
	
        The \textit{tau class} provides a customized design for the table of contents. Each level of the ToC provides a preview of the content and its location in the document. 
		
\section{Tables and figures}

    \subsection{Tables}
	
        Table \ref{tab:table} shows an example table. The \verb|\tabletext{}| is used to add notes to tables easily. 
    		
        \begin{table}[H]
            \centering
            \caption{Astronomical Object Data}
            \label{tab:table}
            \begin{tabular}{ll}
                \toprule
                \textbf{Object} & \textbf{Distance (Light Years)} \\
                \midrule
                Alpha Centauri & 4.37 \\
                Betelgeuse & 642.5 \\
                Andromeda Galaxy & 2.537 million \\
                \bottomrule   
            \end{tabular}
			
            \tabletext{Note: The table contains data of some famous celestial objects.}
			
        \end{table}

    \subsection{Figures}
		
    	Fig. \ref{fig:figure} shows an example figure.
    		
    	\begin{figure}[H]
    		\centering
    		\includegraphics[width=0.75\columnwidth]{Example.pdf}
    		\caption{Example figure obtained from PGFPlots \cite{PFGPlots}.}
    		\label{fig:figure}
    	\end{figure}
		
        Fig. \ref{fig:examplefloat} shows an example of two figures that covers the width of the page. It can be placed at the top or bottom of the page. The space between the figures can also be changed using the \verb|\hspace{Xpt}| command.
		
        \begin{figure*}[tp] % t for position at the top of the current page; b for position at the bottom; p for new page
		\centering
		  \begin{subfigure}[b]{0.38\linewidth} % Fig (a)
			\includegraphics[width=\linewidth]{Example2.pdf}
			\caption{Example left figure.}
			\label{fig:figa}
		\end{subfigure}
			\hspace{20pt}   % Space between the figures
		\begin{subfigure}[b]{0.375\linewidth} % Fig (b)
			\includegraphics[width=\linewidth]{Example3.pdf}
			\caption{Example right figure.}
			\label{fig:figb}
		\end{subfigure}
		\caption{Example figure that covers the width of the page obtained from PGFPlots \cite{PFGPlots}.}
		\label{fig:examplefloat}
	\end{figure*}
		
\section{Tau packages}

    \subsection{Tauenvs}
	
        This template has its own environment package \textit{tauenvs.sty} designed to enhance the presentation of the document. Among these custom environments are \textit{tauenv}, \textit{info} and \textit{note}.
		
        There are two environments which have a predefined title. These can be included by the command \verb|\begin{note}| and \verb|\begin{info}|. All the environments have the same style.
			
        An example using the tau environment is shown below.
		
    	\begin{tauenv}[frametitle=Environment with custom title]
            This is an example of the custom title environment. To add a title type \verb|[frametitle=Your title]| next to the beginning of the environment (as shown in this example).
    	\end{tauenv}
		
        Tauenv is the only environment that you can customize its title. On the other hand, info and note adapt their title to Spanish automatically when this language package is defined.
		
    \subsection{Taubabel}

        In previous versions, we included a package called \textit{taubabel}, which have all the commands that automatically translate from English to Spanish when this language package is defined. 
        
        By default, tau displays its content in English. However, at the beginning of the document you will find a recommendation when writing in Spanish. 
		
        \textit{Note:} You may modify this package if you want to use other language than English or Spanish. This will make easier to translate the document without having to modify the class document.
		
\section{Equation}

    Equation \ref{ec:equation}, shows the Schrödinger equation as an example. 
	\begin{equation} \label{ec:equation}
		\frac{\hbar^2}{2m}\nabla^2\Psi + V(\mathbf{r})\Psi = -i\hbar \frac{\partial\Psi}{\partial t}
	\end{equation} 
    The \textit{amssymb} package was not necessary to include, because stix2 font incorporates mathematical symbols for writing quality equations. In case you choose another font, uncomment this package in tau-class/tau.cls/math packages.
	
    If you want to change the values that adjust the spacing above and below the equations, play with \verb|\setlength{\eqskip}{8pt}| value until the preferred spacing is set.
	
\section{Adding codes}
	
    This class\footnote{Hello there! I am a footnote :)} includes the \textit{listings} package, which offers customized features for adding codes in \LaTeX\ documents specifically for C, C++, \LaTeX\ and Matlab. 
	
    You can customize the format in tau-class/tau.cls/listings style.
	
    \lstinputlisting[caption=Example of Matlab code., language=Matlab,label=code]{example.m}
	
    If line numbering is defined at the beginning of the document, I recommend placing the command \verb|\nolinenumbers| at the start and \verb|\linenumbers| at the end of the code. 
    
    This will temporarily remove line numbering and the code will look better.
	
\section{References}

    The default formatting for references follows the IEEE style. You can modify the style of your references. See appendix for more information.
    
\section{Appendix}

    \subsection{Alternative title}

        You can make the following modification in tau-class/tau.cls/title preferences section to change the position of the title.

\begin{lstlisting}[language=TeX, caption=Alternative title.]
\newcommand{\titlepos}{\centering}
\end{lstlisting}

        This will move the title to the center. 

    \subsection{Info environment}

        An example of the info environment declared in the ‘tauenvs.sty’ package is shown below. Remember that \textit{info} and \textit{note} are the only packages that translate their title (English or Spanish).
		
	\begin{info}
		Small example of info environment.
	\end{info}

    \subsection{Equation skip value}

        With the \verb|\eqskip| command you can change the spacing for equations. The default \textit{eqskip} value is 8pt.

\begin{lstlisting}[language=TeX, caption=Equation skip code.]
\newlength{\eqskip}\setlength{\eqskip}{8pt}
	\expandafter\def\expandafter\normalsize\expandafter{%
		\normalsize%
		\setlength\abovedisplayskip{\eqskip}%
		\setlength\belowdisplayskip{\eqskip}%
		\setlength\abovedisplayshortskip{\eqskip-\baselineskip}%
		\setlength\belowdisplayshortskip{\eqskip}%
	}
\end{lstlisting}
		
    \subsection{References}
		
        In case you require another reference style, you can go to tau-class/tau.cls/biblatex and modify the following.
		
\begin{lstlisting}[language=TeX, caption=References style.]
\RequirePackage[
	backend=biber,
	style=ieee,
	sorting=ynt
]{biblatex}
\end{lstlisting}

        By default, \textit{tau class} has its own .bib for this example, if you want to name your own bib file, change the \textit{addbibresource}.
		
\begin{lstlisting}[language=TeX]
\addbibresource{tau.bib}
\end{lstlisting}

\section{FAQ}

    \subsection*{How do I manage my references?}
        
        To manage your references, I recommend using the tool \href{https://www.scribbr.es/citar/generador/folders/73QOXYsCwMRu4ifQaN65mx/lists/msTfx7GJjIAOUkufbISnA/}{scribbr}. You can simply enter the URL or create your own citation, and then export it to \LaTeX\ using the options in the three-dot menu.
            
        The generated citation can be copied and pasted into \textit{tau.bib}, the file designated for bibliography management. You may rename this file, but if you do, remember to update the \verb|\addbibresource| command in \textit{tau.cls} under the \textit{biblatex} section.
    
        \begin{note}
            Some platforms, such as Google Scholar or scientific journals, provide citations directly in \LaTeX\ format. Therefore, check if there is a ``how to cite this document'' section to streamline the citation process even further.
        \end{note}
    
        Here is an example of a reference code compatible with Bib\TeX.

\begin{verbatim}
@misc{PFGPlots,
    author = {PFGPlots},
    title = {A LaTeX package to create plots.},
    url = {https://pgfplots.sourceforge.net/}
}
\end{verbatim}

        If you have any further questions, you can refer to the following page.

        \begin{itemize}
            \item \href{https://es.overleaf.com/learn/latex/Bibliography_management_with_biblatex}{Bibliography management with biblatex}
        \end{itemize}
            
        An example of an IEEE-formatted bibliography can be found in Fig. \ref{fig:figure}. If you are using a different citation style, such as APA, I recommend using the \verb|\parencite| command to automatically include parentheses around citations.

    \subsection*{What should I do with the example files?}

        If you edit this template, you can remove the example figures, the bibliography entries in \textit{tau.bib}, and the \textit{fibonacci.m} Matlab code without affecting your document.

    \subsection*{How do I convert my document into a column?}

        At the beginning of the document, you will find a \verb|\documentclass| command. By default, it will show \textit{twocolumn}. Simply change this to \textit{onecolumn} and recompile the document. 
        
        If further adjustments are needed, you will have to go to \textit{tau.cls} and navigate to the relevant section (as \textit{geometry package}) to make any other changes or modifications.

    \subsection*{How do I change the paper size?}

        Similarly, you can change the paper size (by default, this class was adapted for A4 size). The following paper sizes are available in \LaTeX:

        \begin{itemize}
            \item letterpaper (11 $\times$ 8.5 in)
            \item legalpaper (14 $\times$ 8.5 in)
            \item executivepaper (10.5 $\times$ 7.25 in)
            \item a4paper (21 $\times$ 29.7 cm)
            \item a5paper (21 $\times$ 14.8 cm)
            \item b5paper (25 $\times$ 17.6 cm)
       \end{itemize}

    \subsection*{How do I place equations easily?}

        For equations, we have two options: inline or on its own line. For inline equations, simply place a dollar sign (\$) at the beginning and end of the equation. However, if you want the equation to be displayed on its own line, you need to use the equation environment.
        
        If you find it challenging to write formulas directly in \LaTeX, you can use text editors like Word. In the equations menu, you can select \LaTeX\ in the conversion section and copy and paste the equation you wrote into one of these two environments.

\section{Contact me}

    You can contact me through these methods.\\
    
    \noindent\faWix\hspace{5pt}\href{https://memonotess1.wixsite.com/memonotess}{https://memonotess1.wixsite.com/memonotess} \\
    \faEnvelope[regular]\hspace{7pt}memo.notess1@gmail.com \\
    \faInstagram\hspace{8pt}memo.notess

\section{Supporting}

    Did you like this class document? Check out our new project the \href{https://es.overleaf.com/latex/templates/rho-class-academic-article-template/bpgjxjjqhtfy}{rho class}, made for complex articles and reports.

    \subsection*{Any contributions are welcome!}
    
        Coffee keeps me awake and helps me create better \LaTeX\ templates. If you wish to support my work, you can do so through PayPal:\\
        \textbf{\url{https://www.paypal.me/GuillermoJimeenez}}.
        
        \begin{center}
            Enjoy writing with tau \LaTeX\ class\hspace{5pt}\faChessKnight 
        \end{center}

%----------------------------------------------------------

\printbibliography

%----------------------------------------------------------

\end{document}